% !TEX root = ../Yan Hao-Dissertation.tex

\chapter{Overall Introduction} \label{chapter1:Introduction---about Fiscal Federalism and Intergovernmental Transfer}

This dissertation delves into the intricate world of intergovernmental transfer payments, a fundamental component of fiscal federalism that underpins the financial interactions between different tiers of government. Structured around four core chapters, this work navigates from a broad analysis of fiscal dynamics to a nuanced examination of specific fiscal behaviors, embodying a journey from the general to the particular, and from upper to lower levels of government.

Chapter 2 examines intergovernmental interactions and transfer payment strategy formation through a dynamic game model, analyzing both government levels' fiscal behaviors. Chapter 3 focuses on how federal transfer payments are influenced by political objectives, highlighting U.S. governance's political nuances. Chapters 4 and 5 delve into local government fiscal responses to transfer payments, with Chapter 4 evaluating their impact on spending and Chapter 5 on tax collection, respectively.


Given the centrality of fiscal transfer payments to this analysis, and their roots in the concept of fiscal federalism, it is imperative to begin by outlining the structure of fiscal federalism and the role of transfer payments within it, with a particular focus on the United States context. This introduction will pave the way for a detailed discussion on the structure of fiscal federalism and the pivotal role played by fiscal transfer payments.

\section{Fiscal Federalism: An Overview}

Fiscal federalism represents a framework for understanding the financial dynamics between various levels of government, addressing the alignment of expenditure responsibilities with revenue sources within hierarchical government structures. In \textcite{musgrave1971economics} and \textcite{oates1972fiscal}'s concept, "theory of fiscal federalism" concerns the division of public-sector functions and finances in a logical way. This concept is foundational for efficiently delivering public goods and services, particularly in nations characterized by complex geographical and administrative layers.

The theories of Hayek, Stigler, and Tiebout form the core of fiscal federalism's theoretical basis. \textcite{hayek2009use} emphasizes the advantage of local governance in identifying and addressing community needs, while \textcite{stigler1998tenable} advocates for ensuring subnational governments' financial autonomy. \textcite{tiebout1956pure} introduces the concept of competitive governance, where citizens 'vote with their feet,' choosing jurisdictions that best meet their preferences for public goods and services. This competition among local governments is argued to improve administrative efficiency, a theory supported by empirical evidence, as illustrated in Table \ref*{Table 1.1}, which shows the variance in tax burden preferences across states.

\subsection{Theoretical Frameworks and Evolutions}
Based on my personal review of the literature, fiscal federalism theory has evolved from its initial focus on the allocation of financial resources and responsibilities (first-generation theory) to examining its broader impacts on economic development, political processes, and governance (second-generation theory).

\subsubsection{First-Generation Theory: Efficiency in Fiscal Structures}

The assessment of fiscal structures' efficiency hinges on established economic principles, notably Pareto efficiency, which mandates resource allocation improvements without detriment to others \parencite{pareto2014manual}. Key economic considerations—externalities, information asymmetry, and incentive compatibility—frame this evaluation. \textcite{oates1972fiscal} discuss the equitable distribution of externalities' costs and benefits within jurisdictions, proposing a model for addressing the "free rider" problem through congruent jurisdictions and beneficiary areas, thereby harmonizing marginal costs and benefits. This principle underlies the tax revenue structure in the U.S., optimizing efficiency and minimizing behavioral distortions in tax collection across jurisdictions, as delineated in Table \ref*{Table 1.2}.

Furthermore, the complexity of information plays a pivotal role in the structural evaluation of fiscal federalism. Drawing on Hayek and Tiebout's foundational work, \textcite{2003Centralized} develop a model emphasizing local governance's superior capacity in public goods provision and the central government's calibrated insensitivity. This model underscores the importance of local knowledge and transparency in government actions, enhancing public goods delivery efficiency through decentralized governance \parencite{martinez2003fiscal,baicker2005spillover}. Such transparency, facilitated by horizontal competition among local governments, fosters an environment where local entities are motivated to enhance their performance.

Incentive compatibility, introduced by \textcite{hurwicz1973design}, emerges as a crucial criterion for fiscal federalism, advocating for system designs that align individual actions with collective welfare. This concept is instrumental in encouraging local governments towards efficient public goods provision, leveraging the alignment of funding mechanisms with organizational incentives to foster diligent and effective governance \parencite{eckstein1958water}. The assumption that local governments aim to maximize fiscal revenues underpins the administrative strategies within fiscal federalism, influencing the allocation of resources and responsibilities \parencite{baretti2002tax,bucovetsky2006efficiency,dahlby2011marginal,jha2000tax}.

Central to first-generation fiscal federalism is the exploration of decentralized structures' efficacy in public goods provision. This theoretical framework endeavors to quantify the effectiveness of fiscal decentralization in enhancing public service efficiency, a pursuit characterized by extensive theoretical exploration.

\subsubsection{Second-Generation Theory: Beyond Efficiency---Fiscal Federalism's Broader Impacts}

Second-generation theories of fiscal federalism delve into its influence beyond the efficiency of public goods provision, examining impacts on economic development \parencite{cai2005does,barro1991economic}, and local government behavior \parencite{jin2005regional}. Highlighting the complex role of fiscal federalism, especially in developing countries \parencite{keen1997fiscal,treisman2002decentralization,bardhan2002decentralization,bucovetsky2005public}, this scholarship identifies three interrelated themes: economic development, political intentions, and local government fiscal conduct.

Tiebout's model serves as the theoretical foundation, yet its assumptions often falter in developing nations where local governments may lack efficiency in public goods provision \parencite{tiebout1956pure}. This literature critically evaluates these assumptions, exploring the nuanced interplay between fiscal federalism and economic growth, factoring in labor and capital mobility \parencite{oates2004essay}. \textcite{faguet2004does} and \textcite{mckinnon1993order} discuss how fiscal federalism can either stimulate or hinder economic development, emphasizing the variances in resource endowment and the potential for decentralization to exacerbate regional disparities \parencite{cai2005does,treisman2002decentralization}.

Furthermore, the influence of fiscal federalism on local government's fiscal strategies, including taxation \parencite{mogues2012external}, expenditure \parencite{hines1995anomalies}, and debt management \parencite{qian1998federalism}, is scrutinized, indicating a broad spectrum of governance and policy implications.

Political dynamics also significantly shape fiscal federalism's architecture, with policies in countries like Canada and Australia reflecting an amalgamation of economic efficiency and political strategy \parencite{oates2005toward}. However, in contexts like Italy, fiscal transfers have intensified interjurisdictional tensions, presenting a contrasting perspective on fiscal federalism's functionality.

The discourse extends to the interactions within fiscal federalism, distinguishing between horizontal and vertical dynamics, and underscoring the pivotal role of vertical interactions between central and subnational entities in shaping fiscal policies and governance outcomes.

In summary, as depicted in Figure \ref*{Figure 1.2}, the original research on fiscal federalism constructed a theoretical framework for efficiently providing public goods. In developed countries, particularly in America, scholars have discovered empirical evidence supporting the advantages of this decentralized fiscal structure. However, in developing countries, fiscal federalism has not worked as effectively, leading to the emergence of second-generation theory. This newer approach focuses on the other side of the coin.


\section{Fiscal Federalism in the United States}

\subsection{Revenue and Responsibilities of Different Levels of Government}
The United States Constitution reserves unspecified rights to state governments, creating a diverse administrative landscape across states due to the country's vast geographical and demographic variety. This diversity results in non-uniform state and local government responsibilities, necessitating a general rather than specific description of administrative structures.

In the American fiscal federalism system, federal, state, and local governments derive income from distinct sources and perform unique roles in public goods provision, with intergovernmental transfers playing a crucial role in financing state and local levels. For fiscal year 2019, federal revenue primarily comprised individual income taxes (50\%), corporate income taxes (7\%), and social insurance or payroll taxes (36\%). State and local revenues significantly rely on intergovernmental transfers, averaging over 30\%, supplemented by sales and property taxes.\footnote[1]{Data Source: The Department of the Treasury and the Bureau of the Fiscal Service}

The expenditure framework reveals distinct roles across government tiers in public goods and services delivery, after excluding debt interest. The federal government funds income security, social security, health, national defense, and infrastructure, among others. State and local governments, meanwhile, focus on public welfare, education, health, and community development.\footnote[2]{Data Source: The Department of the Treasury and the Bureau of the Fiscal Service} Figure\ref*{Figure 1.3} and Figure\ref*{Figure A.1} in Appendix A illustrate the revenue and expenditure patterns, indicating stability with notable annual fluctuations.

\subsection{Interaction between National and Subnational Governments}

The U.S. engages in public goods provision at the subnational level primarily through joint provision and intergovernmental transfers. The federal government influences state and local fiscal decisions through grants-in-aid (GIA) and intergovernmental transfers (IGT), amounting to nearly \$700 billion annually, or about 20\% of federal revenue \parencite{dilger2015federal}. These mechanisms aim to leverage state and local governments' proximity to constituents, fostering efficiency, spatial customization, and democratic engagement \parencite{musgrave1997devolution}.

Grants-in-aid in the U.S. vary by restriction level and administrative procedures, categorized into categorical grants, block grants, and general revenue sharing grants, with categorical grants dominating in number and fiscal outlay \parencite{dilger2015federal}. Administrative procedures differentiate grants into competitive, formula-based, and reimbursement grants, each with distinct allocation criteria and objectives.

Joint provision, in contrast, involves direct public goods delivery by the national government, particularly in infrastructure projects. This mechanism simplifies revenue collection and goods provision, bypassing intergovernmental transfers.

This section elucidates the fiscal federalism structure in the United States, highlighting the nuanced revenue and responsibility distribution across government levels and the complex interaction mechanisms that facilitate public goods provision within this framework.

\section{Impetus and aiming of the Dissertation}

The impetus for this dissertation arises from the increasingly complex interplay between central and local governments within fiscal federalism frameworks, particularly under asymmetric settings. Fiscal federalism, a cornerstone of public administration and economics, explores the allocation of financial responsibilities and resources among different levels of government. This dissertation is driven by a desire to deepen the understanding of how central and local government interactions, shaped by fiscal federalism, affect fiscal behavior and policy outcomes. Despite a wealth of theoretical and empirical studies, gaps remain in our comprehension of the dynamic interactions and the mechanisms behind intergovernmental transfers.

The primary aim of this dissertation is to construct a nuanced theoretical analysis supplemented by empirical evidence that elucidates the strategic decision-making processes and outcomes of fiscal interactions between central and local governments. This study seeks to address the following critical questions:
\begin{itemize}
    \item How do central and local governments strategize their fiscal policies within the context of fiscal federalism, and what implications do these strategies have for public goods provision?
\end{itemize}

\begin{itemize}
    \item How do political dynamics, such as party control and alignment influence the distribution and utilization of intergovernmental grants?
\end{itemize}

\begin{itemize}
    \item What are the effects of different intergovernmental transfer mechanisms on the fiscal behavior of local governments, particularly in terms of public goods provision and tax collection effort?
\end{itemize}
By answering these questions, the dissertation aims to offer a comprehensive understanding of fiscal federalism's practical applications and theoretical underpinnings, focusing on the United States as a case.

\section{General Overview of Research Approach}

This dissertation employs a comprehensive research approach that intertwines theoretical analysis with empirical evidence to explore the complex interplay between central and subnational government fiscal behaviors under fiscal federalism. The objective is to delineate a nuanced understanding of fiscal dynamics across governmental tiers, leveraging a multi-methodological framework that encompasses both deductive and inductive reasoning. This section provides an overview of the research methodology, highlighting the integration of theoretical models and empirical analysis to address the dissertation's central inquiries.

\subsection{Theoretical Analysis}
The theoretical foundation of this dissertation is anchored in a diverse array of analytical frameworks beyond the conventional application of game theory. While game theory provides a structured mechanism to predict strategic interactions between governmental entities, this study extends its theoretical exploration to include mathematical models and economic theories that offer deeper insights into fiscal phenomena.
\begin{itemize}
    \item Game Theory Models
\end{itemize}

Central to the analysis are dynamic game theory models that simulate the strategic decision-making processes between the national and subnational governments. These models illuminate the conditions under which different fiscal policies are adopted and their potential outcomes on public goods provision. The game theory approach is instrumental in understanding the strategic underpinnings of fiscal interactions and the implications of various intergovernmental transfer mechanisms.

\begin{itemize}
    \item Mathematical Conjectures and Models
\end{itemize}

Beyond game theory, the dissertation employs mathematical conjectures and models to derive theoretical implications of fiscal policies. This includes the utilization of Ramsey models in examining the effects of grants on subnational governments' revenue collection efforts and the analysis of public expenditure behavior through mathematical expressions of the flypaper effect. These mathematical analyses provide a rigorous foundation to theorize the impact of fiscal federalism on government behaviors and policy outcomes.

\begin{itemize}
    \item Economic Theories and Frameworks
\end{itemize}

The theoretical analysis is further enriched by the application of economic theories and frameworks that contextualize the mathematical models and game theory simulations within the broader discourse of fiscal federalism. This includes theories of fiscal decentralization, public goods provision, and fiscal illusion. These frameworks offer a conceptual lens through which the empirical findings can be interpreted, bridging the gap between abstract theoretical deductions and real-world fiscal behaviors.



\subsection{Empirical Analysis}
To complement the theoretical analysis, the dissertation incorporates a robust empirical investigation aimed at validating the theoretical models and uncovering real-world evidence of the hypothesized fiscal dynamics. This entails:

\begin{itemize}
    \item Panel Data Analysis
\end{itemize}

Utilizing panel data over several years, the dissertation examines the influence of political party control on intergovernmental grants in the United States. This analysis seeks to empirically validate the theoretical predictions concerning the political dimensions of fiscal federalism.

\begin{itemize}
    \item Case Studies and Comparative Analysis
\end{itemize}

By reviewing specific instances of fiscal federalism in action, the study provides empirical insights into the practical implications of theoretical models. This includes examining the flypaper effect and its influence on local governments' spending behavior.

\begin{itemize}
    \item Quantitative Methods
\end{itemize}

The empirical counterpart of this dissertation employs a synthesis of advanced analytical techniques, collectively referred to here as "Quantitative Analytical Methods." This classification encompasses both Principal Component Analysis (PCA) and Kalman Filter estimation, among others, serving as the empirical foundation upon which theoretical hypotheses are tested against real-world data.

PCA analysis, used predominantly in the examination of political party control on intergovernmental grants, facilitates the distillation of complex datasets into principal components, thus enabling a clearer understanding of underlying patterns and impacts. Similarly, the Kalman Filter, applied in the study of grants' restriction on subnational governments' revenue collection effort, offers a sophisticated method for estimating time-varying parameters within dynamic systems. Together, these Quantitative Analytical Methods bridge the gap between theoretical constructs and their empirical manifestations, providing a rigorous validation of the hypotheses derived from the theoretical analysis.

This integrated approach, combining theoretical analysis with empirical evidence, positions the dissertation to contribute significantly to the understanding of intergovernmental transfer under fiscal federalism system. By bridging the gap between theory and practice, it offers valuable insights into the design of fiscal policies that optimize the balance between national objectives and local autonomy, promoting a more equitable distribution of public resources.

\section{Significance of the Dissertation}

The significance of this dissertation extends beyond academic contributions to the fields of public administration and economics. It provides a critical examination of fiscal federalism's role in shaping the efficiency and equity of public goods provision across different government levels. By integrating game theory models with empirical analysis, this study offers novel insights into the strategic behaviors of central and local governments, highlighting the impact of these behaviors on fiscal policies and outcomes.

For policymakers, the findings of this dissertation offer valuable guidance in designing intergovernmental transfer mechanisms that promote efficient and equitable public goods provision. It sheds light on the importance of considering both financial and strategic dimensions in fiscal policy-making, thereby informing the development of more effective intergovernmental fiscal relations.

Furthermore, this research contributes to the broader discourse on fiscal federalism by challenging and expanding upon existing theories, particularly in the context of asymmetric government settings. It brings to the forefront the critical role of political dynamics in fiscal federalism, emphasizing the need for a holistic approach that accounts for both economic efficiency and political feasibility in the design and implementation of fiscal policies.

In summary, the dissertation stands as a significant endeavor in understanding and improving the intricate mechanisms of fiscal federalism, with implications for both theory and practice in public fiscal management and governance.


\section{Overview of Subsequent Chapters}

This dissertation aims to dissect the multifaceted questions of intergovernmental transfer embodied in fiscal federalism system in United States, with the focus on the interaction between federal government and state government. Given the dual-layered government interaction inherent in this framework, the analysis bifurcates into examining both the central and subnational levels.

Central level inquiries delve into the national government's engagement in subnational public goods provision—choosing between non-involvement, intergovernmental transfers, or direct provision—and the mechanics behind intergovernmental transfer distribution. Subnational level analysis focuses on how local governments respond to federal decisions, specifically regarding expenditure and revenue collection behaviors.

Chapters dedicated to central government questions explore the decision-making processes for engagement methods and intergovernmental transfer mechanisms. Subsequent chapters address subnational reactions, analyzing spending behaviors post-transfer receipt and revenue collection strategies supported by federal grants. Each query is bolstered by theoretical frameworks and empirical validations, structured as a $2\times2$ matrix for clarity, detailed in Table \ref*{Table 1.5}.

Chapter 2 sets the stage by exploring the overarching issues of intergovernmental interactions, focusing on the mechanisms that shape transfer payment strategies. Employing a fully informed dynamic game model, it elucidates the fiscal behaviors of both upper and lower levels of government, offering insights into the formation of transfer payment mechanisms.

Chapter 3 narrows the focus to the upper government level, specifically examining how transfer payment decisions at the federal level in the United States are influenced by political party objectives, reflecting the political realities of American governance.

Chapters 4 and 5 further refine the scope by concentrating on local government fiscal behavior, divided into public spending and revenue collection actions. Chapter 4 synthesizes existing literature to assess the impact of transfer payments on local government expenditure behaviors, while Chapter 5 investigates the influence of upper-level government transfer payments on the tax collection practices of local governments.

By examining the intricacies of fiscal federalism, this research directly impacts public administration by providing empirical evidence and theoretical insights into how fiscal decentralization influences governmental efficiency and accountability. It sheds light on the critical role of intergovernmental transfers in aligning financial resources with public service demands, revealing the impact of political dynamics on fiscal decisions. This investigation enhances understanding of fiscal policy's role in promoting equitable and efficient public services, offering valuable strategies for public administrators to optimize resource allocation and foster more responsive and responsible governance structures.
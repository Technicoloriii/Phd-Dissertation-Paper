\chapter{Summary, Thoughts and Rethink}

Within the vast landscape of fiscal systems, understanding the dynamics between resource collection and reallocation is crucial, as it directly ties to the provision of public goods. This discourse inevitably introduces a pivotal concept: the "tradeoff," which underlies fiscal system designs, embodying the necessity to balance divergent goals and constraints. Fiscal federalism, a structure widely embraced across the globe, brings the principles of centralization and decentralization into sharp relief, particularly within the realm of intergovernmental transfers. These mechanisms showcase a constant negotiation between the need for centralized coordination and oversight to achieve national objectives and the decentralized approach that values regional autonomy and tailored responses to local needs. This chapter delves deeper into how intergovernmental transfers encapsulate these tradeoffs, reflecting broader themes of fiscal policy and governance. It seeks to distill and expand upon insights scattered throughout the investigation into a cohesive summary, thus framing the thesis within the ongoing dialogue between centralization and decentralization in public administration. In this chapter, I want to extract some thoughts that spread all over the intergovernmental transfer investigation, which also accounts to a summary of this thesis.

\section{Overall Interests and Partial Interests}

One of the intrinsic contradictions within intergovernmental transfer design, is the balancing act between overall interests and partial interests. This dichotomy is not merely a conceptual framework but a practical reality that affects decision-making at various levels of government—national and subnational. The inherent tradeoff between these interests is a reflection of the complex interplay between the desire for collective welfare and the pursuit of localized benefits.

Intergovernmental transfer policy, by design, incorporates at least two layers of government: the national government and multiple layers of subnational governments, each with its domain of fiscal responsibilities and autonomy. The national government is tasked with considering the benefits for all regions, aiming to optimize the welfare of the country as a whole. In contrast, subnational governments, such as states or provinces, focus on their regions' specific needs and priorities.

This divergence in focus leads to a fundamental tradeoff in fiscal policies and resource allocation. National governments, aiming for broader economic stability and equitable development, might prioritize policies that do not always align with the immediate interests or preferences of subnational governments. For instance, a national government might implement general transfer payments to ensure that all regions have the necessary resources to provide basic public services. However, these allocations may not always reflect the unique needs or priorities of individual regions, leading to disparities in satisfaction and perceptions of fairness.

The dynamic complete game model discussed in Chapter 2 underscores this tradeoff, illustrating how national governments utilize general transfer payments and joint provisions to reconcile these conflicting interests. The model's equilibrium outcomes suggest that while general transfers and joint provisions are essential tools for addressing overall interests, they might not universally be welcomed by all subnational regions. This scenario encapsulates the essence of the tradeoff between overall and partial interests: the pursuit of collective welfare might necessitate compromises that do not fully satisfy the localized ambitions of individual regions.

Moreover, this tradeoff extends beyond fiscal transfers to encompass the broader considerations of externality management and public goods provision. The principle that "the whole is more than the sum of its parts" finds practical expression in how externalities are addressed within the fiscal federalism framework\parencite{1988Foreign, Branstetter2001Are, 2011Procurement}. National governments are often in a position to internalize externalities that transcend regional boundaries, such as environmental pollution, interstate infrastructure projects, and national education standards. These efforts, aimed at optimizing the overall welfare, might impose constraints or redirect resources in ways that do not align with the immediate preferences of subnational governments.

In summary, the tradeoff between overall interests and partial interests within fiscal federalism is a reflection of the inherent tensions between collective welfare and regional autonomy. This tradeoff is not merely a theoretical construct but a practical reality that influences policy formulation, resource allocation, and the pursuit of equity and efficiency in public goods provision. As this dissertation has illustrated through models and empirical analysis, navigating this tradeoff requires a nuanced understanding of the fiscal federalism framework and a commitment to balancing diverse interests in the pursuit of shared prosperity.


\section{Efficiency and Equity}

The balancing act between efficiency and equity represents one of the most profound dilemmas in the realm of intergovernmental transfer design. This tradeoff is not only a reflection of differing policy objectives but also a manifestation of the underlying values that guide fiscal policy decisions. Efficiency, in this context, refers to the optimal allocation of resources to maximize the economic output, whereas equity is concerned with the fair distribution of resources among different regions or population groups.

Intergovernmental transfers, as a critical component of fiscal federalism, are designed to address disparities across jurisdictions by reallocating financial resources from wealthier to less wealthy areas. However, the pursuit of equity through these transfers often encounters the challenge of maintaining economic efficiency. The inherent tension arises because policies aimed at redistributing resources to achieve a more equitable outcome may inadvertently affect the incentives for local governments and individuals, potentially leading to inefficiencies in resource allocation and utilization.

One aspect of this tradeoff is visible in the design of transfer mechanisms themselves. General transfer payments, aimed at supporting a broad range of expenditures without specific conditions, may promote equity by providing underfunded regions with necessary resources. However, without targeted conditions, these transfers may not always be used in the most efficient manner, as recipient governments have broad discretion over their use. This discretion can lead to inefficiencies if the funds are not allocated to areas of greatest need or potential economic impact.

Conversely, categorical or conditional transfers, which specify the purposes for which funds must be used, aim to enhance efficiency by directing resources to predetermined priorities such as infrastructure development or education. While potentially more efficient, these targeted transfers may not always address the most pressing needs of a region, thereby raising questions about equity. For example, a categorical grant for highway construction may benefit regions with significant transportation needs but do little for areas where healthcare or education requires more urgent attention.

The empirical evidence and theoretical models discussed throughout the dissertation illustrate these tradeoffs in practice. For instance, the dynamic complete game model presented in Chapter 2 highlights how national and subnational governments negotiate the allocation of resources, with each level of government weighing the benefits of efficiency and equity differently. Furthermore, the empirical analysis of transfer payments in the United States from 2000 to 2019 provides insight into how these tradeoffs manifest in real-world fiscal policy decisions.

Ultimately, the tradeoff between efficiency and equity in intergovernmental transfers is a reflection of the broader challenge of balancing competing objectives within fiscal federalism. While efficiency is crucial for maximizing the impact of public expenditures, equity is essential for ensuring that all citizens, regardless of where they live, have access to public services and opportunities for economic advancement. Navigating this tradeoff requires careful policy design, transparent decision-making processes, and ongoing evaluation to adjust strategies in response to changing needs and priorities.



\section{Long Term Interest and Short Term Interest}
The tension between short-term gains and long-term sustainability represents a crucial tradeoff in the realm of fiscal federalism, particularly in the design and implementation of intergovernmental transfers. This tradeoff is at the heart of policy decisions, where immediate benefits are often weighed against potential future costs, influencing the strategic allocation of resources and the formulation of fiscal policies.

Intergovernmental transfers, as mechanisms for redistributing resources among different government layers, embody this tradeoff through their impact on fiscal behaviors, incentives for economic development, and the overall stability of public finances. The choice between prioritizing short-term financial relief or investing in long-term developmental objectives poses significant implications for economic equity and efficiency.

One manifestation of this tradeoff is the strategic use of categorical transfers. Categorical transfers are often designed to address specific policy objectives, such as infrastructure development or education funding. While these transfers can stimulate immediate spending in targeted areas, their restrictive nature may limit the recipient government's ability to allocate resources toward long-term strategic goals. This scenario underscores the tradeoff between achieving immediate policy outcomes and maintaining fiscal flexibility to address future challenges.

Furthermore, the issuance of bonds by national governments to support categorical transfers introduces another layer to the short-term versus long-term debate. While bonds can provide immediate funding for critical projects, they also impose future debt obligations, transferring the financial burden to future generations. This dynamic reflects a critical tradeoff: the decision to leverage future financial stability for immediate developmental gains.

The impact of this tradeoff extends beyond fiscal policy design to influence the broader economic landscape. Short-term fiscal interventions, while effective in addressing immediate challenges, may not always align with long-term sustainability goals. For example, short-term economic stimuli, though beneficial in revitalizing an ailing economy, may exacerbate long-term fiscal imbalances if not carefully managed. This interplay between immediate benefits and future costs requires a delicate balancing act, ensuring that short-term decisions do not undermine long-term fiscal health and economic stability.

In sum, the tradeoff between short-term interests and long-term sustainability is a defining feature of fiscal federalism, with profound implications for policy formulation, resource allocation, and intergovernmental relations. As highlighted throughout this dissertation, understanding and navigating this tradeoff is essential for achieving balanced and sustainable fiscal outcomes. Policymakers must therefore carefully consider the long-term implications of their decisions, striving to balance immediate needs with future aspirations in the pursuit of fiscal stability and economic prosperity.

% \section{Centralization and Decentralization}

% The concepts of centralization and decentralization are pivotal in understanding the architectural dynamics of intergovernmental transfers within fiscal federalism. These terms transcend mere administrative structures to encapsulate the essence of governance, resource allocation, and policy effectiveness. Centralization refers to the concentration of decision-making authority at higher levels of government, typically aiming for uniformity and coordination across broader territories. In contrast, decentralization represents the devolution of powers to lower levels of government, promoting local autonomy, responsiveness, and tailored governance solutions.

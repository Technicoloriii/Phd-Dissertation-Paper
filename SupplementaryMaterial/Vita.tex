% Place your vita below.

% First of all, I would like to give my heartfelt thanks to all the people who have ever helped me in this paper.

% My sincere and hearty thanks and appreciations go firstly to my supervisor, Mr. Stalebrink, whose suggestions and encouragement have given me much insight into these translation studies. It has been a great privilege and joy to study under his guidance and supervision. Furthermore, it is my honor to benefit from his personality and diligence, which I will treasure my whole life. My gratitude to him knows no bounds.

% I am also extremely grateful to all my friends and classmates who have kindly provided me assistance and companionship in the course of preparing this paper.

% In addition, many thanks go to my family for their unfailing love and unwavering support.

% Finally, I am really grateful to all those who devote much time to reading this thesis and give me much advice, which will benefit me in my later study.

I remember it was an afternoon during exam week. Typically, I would take a nap in the afternoon, but since coming to the United States two years ago, this habit has been sporadic. Coupled with the fact that it was exam week in the winter semester, I, like anyone striving for a decent GPA, spent the entire day in the Sheridan library in Johns Hopkins DC campus, preparing for exams in microeconometrics and Bayesian econometrics.

The view outside the window is so stunning that it's mesmerizing. The streets of DC are clean, yet covered in fallen leaves, flanked by buildings that are several decades, even centuries old. Each building has a stone plaque at its entrance, detailing the stories behind them. The fall season in the Eastern United States already carries a chill, and looking through the window, with raindrops patterning against the glass, I can imagine the cold outside, making the warmth of the fourth floor of the library especially precious.

Lunchtime sandwiches arrived promptly, but within 30 minutes, drowsiness started creeping in. Surrounded by disheveled students slumped over their desks, I realized my brain was in no condition to delve into the intricacies of exogenous identification or any kind of math at that moment. So, I turned to the supplementary reading the professor had recommended. I randomly picked an article and planned to skim through the abstract, just for the sake of it. I can't recall the exact title of that article, but I vaguely remember it was about the theoretical analysis and VAR testing of the effects of fiscal stimulus on economic growth. Surprisingly, I found myself completely engrossed in the article's summary of related literature and its presentation of various perspectives and logical arguments. What's more, out of sheer interest, I ended up tracking down the original articles referenced. Consequently, what was initially meant to be an afternoon dedicated to revising for final exams irrepressibly transformed into several hours of diving into academic literature.

At that time, understanding academic articles was quite challenging for me as a non-native English-speaking graduate student. I struggled through reading three or four articles with only partial comprehension, and before I knew it, the day turned into night. I realized that continuing in this manner would seriously jeopardize my performance in the final exams. It was at that moment, as I looked at the half-darkened sky outside the window, that a thought popped into my mind for the first time: What is the actual explain of this problem? Should I pursue a Ph.D.?

It's been four years now, and I've long forgotten what that initial question was that sparked my interest in pursuing a Ph.D. The research topics during my doctoral studies have also evolved. At this moment, I am in the process of writing the acknowledgments section of my thesis, marking the nearing conclusion of my Ph.D. journey. Looking back over these past four years, I dare not claim any significant scholarly achievements as a researcher, but I have indeed persevered through each challenging yet fulfilling day. Certainly, apart from the academic endeavors during these four years in Pennsylvania, there have been occasional moments of joy, along with the enjoyment of life, whether through small breakthroughs or savoring moments with a few close friends, especially Yulin Xu and Tiangeng Lu in the wilderness of Pennsylvania, occasionally finding ways to improve life by enjoying good food once upon a time, typically handmade BBQ or that Vietnamese noodle house named "Little Saigon". During the year-long period of staying at home to avoid the COVID-19 pandemic, the rice noodles from Little Saigon were a rare delight for me.

I am deeply grateful to Dr. Odd Stalebrink, from whom I've learned a great deal, not only in academic guidance but also in navigating the challenges of a strenuous doctoral journey. Also, to Dr. Kim Younhee, who was my professor in Public Administration, a lovely and amiable lady. I am grateful to Dr. Mallinson, a member of my dissertation committee. Additionally, my professor in Macroeconomics, Dr. David Argente, who later moved to Yale University for teaching. I appreciate his dedication and wish him all the best.

Additionally, I am deeply grateful to my mom and dad for providing me with all the support, both material and emotional that parents can offer. They've been doing this for thirty years, especially during my doctoral studies, where it felt like they were going through the degree alongside me. And to my girlfriend, thank you for your support, especially given the distance of over ten thousand kilometers that separates us on opposite ends of this planet.

The four years at Pennsylvania State University may not be long in the grand scheme of life, but they are moments I will forever cherish. In the days and nights of my future, I fear I will often find myself reminiscing about the snowy nights where I furrowed my brow in contemplation. Each of these moments will serve as a driving force for my future endeavors and will shape me. There's a Chinese saying, 'at thirty, one stands firm.' In two days, I will turn thirty. In this section of gratitude, aside from expressing my thanks, I also consider it as my own wish list. I hope that in the forthcoming phase of my life, I can have a happy and fulfilling family and pursue a career that I am passionate about and dedicated to.

One should strive continuously to strengthen himself.
\\
\\
\\
\\
\\
\\
Yan Hao
\\
11/07/2023
\\
Two days before my 30th birthday
\\
Chengdu,Sichuan,China.
% Place abstract below.

This dissertation undertakes a comprehensive analysis of the intricate interplay between central and subnational governmental entities within the framework of fiscal federalism. This investigation is structured around the exploration of three distinct inquiries pertinent to both national and subnational administrations. These three inquiries are systematically delineated into two hierarchical levels and subsequently deliberated upon across three dedicated chapters.

At the national level, this study advances existing literature by introducing a novel theoretical framework concerning the decision-making mechanism behind the provision of public goods. This framework is formulated as a dynamic game with complete information involving three distinct actors. By employing a backward induction methodology to solve the game, the outcomes yield distinct theoretical implications, notably diverging from prior investigations due to the model's modified structure. A significant proposition posited in this study asserts that the motivation of subnational governments to augment the supply of public goods is subdued once a specific threshold of public goods provision is achieved. Consequently, supplies from the national level, whether through collaborative provisioning, general transfers, or categorical transfers, garner greater favor from subnational entities than previously hypothesized. Additionally, the influence of political bias emerges as a determinant in decision-making, functioning as a multiplier affecting the perceived ideal policy gap. The conjecture put forth suggests that the national government might be inclined to tolerate a greater policy gap when effectuating transfers to subnational governments that share the same political affiliation. Furthermore, to validate the theoretical assertions made, an empirical analysis is conducted utilizing a fixed effect regression model. This model is applied to longitudinal data obtained from various American states, aiming to establish empirical evidence in support of the theoretical underpinnings outlined in this study.

At the subnational level, a meticulous examination was undertaken to ascertain the impact of intergovernmental transfers on the expenditure patterns and revenue collection strategies of subnational governments. Regarding expenditure behavior, the existing literature concerning the elucidation of the "flypaper effect" was synthesized. I set up a Ramsey planner’s problem and generate an analytical solution representation delineating the extent of the flypaper effect under both distortionary and non-distortionary tax regimes. The implications arising from this analysis manifest in two principal dimensions. Firstly, the collection of distortionary taxes exhibits a discernible price effect, thereby serving as a primary catalyst for the observed flypaper effect phenomenon. Secondly, it is posited that the magnitude of the flypaper effect is amplified when public goods demonstrate a lack of substitutability with private goods. These theoretical inferences are corroborated by the empirical findings from the collection of pertinent literature previously amassed in the course of prior research endeavors.

In contrast to the scrutiny afforded to the impact of intergovernmental transfers on the expenditure facet, the investigation into their effect on the revenue aspect remains a less explored area. Therefore, this study establishes a theoretical framework aimed at elucidating the ramifications of intergovernmental transfers on the tax collection endeavors of subnational governments. The derived conclusions indicate that when governments place a greater emphasis on productive expenditures, an augmentation in general transfers is likely to trigger a downturn in tax effort. Conversely, an increase in categorical transfers is anticipated to correspond with heightened tax collection efforts. Building on these theoretical inferences, empirical substantiation is sought by means of a Difference-in-Difference analysis, employing the paradigm of fuel taxation in the United States as a case study. Through this empirical investigation, the study endeavors to provide tangible evidence supporting the theoretical propositions posited.

To summarize, the thesis is a comprehensive investigation of different level governments’  fiscal behavior under fiscal federalism including both theoretical inference and empirical investigation. This thesis advanced some existing model and narrow down the gaps in prior research.

Key words: fiscal federalism, asymmetric game theory analysis, longitudinal study


\vspace{-0.3in}
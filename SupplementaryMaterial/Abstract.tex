% Place abstract below.

The allocation of funding and administrative responsibilities between central and subnational governments remains a challenging issue in the field of public administration and public finance for many sovereign states. Fiscal federalism, a framework first introduced by Musgrave, a German-born American economist, has been extensively studied as a means of addressing this issue in both unitary and federal systems. Fiscal federalism aims to determine the optimal division of responsibilities, including finances, between central and subnational governments to efficiently supply public goods and services and achieve potential political objectives. However, determining the appropriate allocation of responsibilities is complicated due to divergent subjective perspectives on the role of government. As a result, fiscal federalism research generally renders no judgment on the proper level of total government intervention or what types of services governments should provide. The research focuses instead on how responsibilities are assigned across multiple layers of government once policymakers have decided to implement a given policy, and what trade-offs may be involved in administration process.

This paper examines the interaction of central and subnational governments under fiscal federalism through an investigation of 3 distinct questions related to central and subnational governments. 3 questions are divided into 2 levels. On national level, I advanced the theory about the intergovernmental transfer decision making mechanism and empirically tested it. On subnational level, I investigated the effect of IGT on subnational governments' spending behavior and revenue collection behavior. Through a comprehensive analysis both theoretically and empirically, a comprehensive understanding of the interactions between different level of governments under fiscal federalism is generated, including an explanation of the behavior mechanism. Specifically, both qualitative and quantitative analysis of fiscal structure design and administrative reality, with a focus on the United States, sheds light on the role of political impact in deciding the grants distribution when I investigate the question on central level. What's more, by making reasonable assumptions, I generate a arithmetic solution of the flypaper effect and understand the spending reaction of subantional governments.  Besides, using game theory tools, this study also provides a theoretical explanation for the differences between administrative reaction of subnational governments and theoretical design under an asymmetric setting and this study conducts empirical tests to statistically support this theoretical inferences.

Key words: fiscal federalism, asymmetric game theory analysis, longitudinal study


\vspace{-0.3in}
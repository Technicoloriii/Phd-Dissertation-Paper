% Place abstract below.

This dissertation undertakes a comprehensive analysis of the intricate interplay between central and subnational governmental entities within the framework of fiscal federalism. This investigation is structured around the exploration of three distinct inquiries pertinent to both national and subnational administrations. These three inquiries are systematically delineated into two hierarchical levels and subsequently deliberated upon across three dedicated chapters. The findings from these studies offer a comprehensive insight into the complexities of fiscal interactions, advancing our understanding of how intergovernmental transfer mechanisms and governmental strategic behaviors influence the efficiency and effectiveness of public service delivery across different governmental tiers. This enhanced perspective contributes significantly to the field by detailing the conditions under which fiscal policies can be designed to optimize the balance between national objectives and local autonomy, thereby promoting a more equitable distribution of public resources.

% At the national level, this study advances existing literature by introducing a novel theoretical framework concerning the decision-making mechanism behind the provision of public goods, introduced in chapter 2. Chapter 2 presents a nuanced exploration of intergovernmental transfer decisions through a dynamic game theory model, focusing on the interactions between central and subnational governments in the provision of public goods. By employing backward induction, the study uncovers the strategic considerations underlying these fiscal interactions, revealing how different transfer mechanisms—no provision, joint provision, general transfer, and categorical transfer—affect public goods provision and utility outcomes at both the national and subnational levels.

% The results highlight several key findings: first, the no provision game serves as a baseline, indicating that subnational governments with higher demand, tax collection efficiency, and larger tax bases tend to supply more public goods. Second, contrary to prior assertions, national government provision in the joint provision game acts as a pure squeeze-out, reducing subnational public goods provision. Third, in the general transfer game, subnational governments adjust their provision based on the balance between income effects and squeeze-out effects, illustrating the complex impact of national-level decisions on local public goods provision. Fourth, the study introduces a critical analysis of the categorical transfer game, demonstrating how matching ratios influence subnational government decisions, with a higher price for welfare-oriented goods leading to greater utility from increased public goods provision.

% Furthermore, chapter 2 discusses the implications of these findings on the choices of intergovernmental transfers, emphasizing the strategic behavior of national governments to ensure the acceptance of their preferred transfer mechanism and the conditions under which subnational governments are likely to accept categorical transfers.

% The research in chapter 2 contributes to the broader understanding of fiscal federalism by providing empirical and theoretical insights into the strategic interactions between different levels of government in the context of intergovernmental transfers. It underscores the importance of considering both the financial and strategic dimensions of fiscal policies to enhance the efficiency and equity of public goods provision across jurisdictions.

Chapter 2 introduces a dynamic game theory model to analyze the strategic interactions between central and subnational governments in the allocation of public goods. Employing backward induction, it reveals how different intergovernmental transfer mechanisms—no provision, joint provision, general transfer, and categorical transfer—affect public goods provision and utility outcomes at both the national and subnational levels. Key findings illustrate the strategic behaviors of governments, demonstrating the conditions under which subnational entities accept categorical transfers. This chapter contributes to understanding the decision-making process in fiscal federalism, highlighting the complex balance between national objectives and local needs.


% In chapter 3, This dissertation explores the complex dynamics of political party control and its influence on intergovernmental grants in the United States, employing a panel data analysis over a 19-year period. Through a detailed examination utilizing Principal Components Analysis (PCA) and regression models, the study investigates how political alignment between state and federal governments affects the distribution of intergovernmental transfers. The PCA reveals that states governed by the same political party exhibit similar economic and social characteristics, forming distinct clusters based on party alignment. This clustering suggests a significant relationship between party control and state characteristics, which in turn influences grant allocation.

% The regression analysis further uncovers that political alignment significantly impacts the amount of intergovernmental transfers received by states. Specifically, states with unified government control, where both the administrative and legislative branches at the state and federal levels are controlled by the Republican Party, tend to experience a reduction in grants compared to states under bipartisan control or Democratic control. The findings challenge the hypothesis that swing states receive more grants, showing instead that in non-election years, federal governments are hesitant to allocate excessive resources to states controlled by opposing parties.

% Chapter 3 contributes to the understanding of fiscal federalism by highlighting the role of political factors in the allocation of federal grants. It provides evidence that political alignment between state and federal governments can influence the distribution of intergovernmental transfers, suggesting that political considerations are embedded within the fiscal policy-making process. This research offers insights into the strategic behavior of political parties in legislative bargaining over grant distribution, with implications for policy-making and the study of federalism and intergovernmental relations.

Chapter 3 examines the influence of political party control on intergovernmental grants in the United States using panel data analysis over 19 years. It uncovers how political alignment between state and federal governments impacts the distribution of intergovernmental transfers. Through Principal Components Analysis and regression models, the study finds that states with unified political control tend to receive fewer grants, challenging the hypothesis that swing states are favored in grant allocation. This analysis provides insights into the role of political factors in fiscal federalism.


% In chapter 4, I summarized the literature on flypaper effect systematically to better understand the effect of intergovernmental transfer on subnational governments' spending behavior.

Chapter 4 delves into the "flypaper effect" and its implications on local governments' spending behavior following intergovernmental transfers. By reviewing literature and employing a mathematical model, it systematically analyzes how these transfers influence local expenditure decisions. The chapter contributes to the fiscal federalism discourse by offering a comprehensive review of the flypaper effect, demonstrating its significance in understanding local governments' fiscal responses to federal grants.


% Chapter 5 offers a comprehensive analysis of how intergovernmental grants influence the tax collection efforts of subnational governments, emphasizing the distinction between general and specific grants in the provision of public goods and services. By employing a nuanced Ramsey model, the study examines the potential shifts in tax effort induced by these transfers and introduces an innovative panel data model using the Kalman filter to estimate tax effort more accurately. This methodological approach addresses the inherent challenges in measuring tax effort and distinguishes between the effects of different types of grants on local tax collection practices. The research contributes to the literature by providing deeper insights into the dynamics of fiscal federalism, highlighting how specific transfer payments can either incentivize or disincentivize local tax effort, depending on their structure and implementation. Through its findings, the study aims to inform policymakers on the design of intergovernmental transfers that can enhance the fiscal autonomy and efficiency of subnational governments, thereby improving public service delivery and fiscal sustainability at the local level.

Focusing on the impact of grants on subnational governments' revenue collection efforts, Chapter 5 employs a nuanced Ramsey model and Kalman filter estimation to examine how general and categorical grants affect local tax efforts. The study distinguishes between the incentives provided by different types of grants, offering empirical insights into the dynamics of fiscal federalism. It suggests that specific transfer payments can either incentivize or disincentivize local tax effort, highlighting the importance of grant design in enhancing fiscal autonomy and efficiency at the local level.

\ \hfill \

\noindent \textit{Key words: intergovernmental transfer, game theory inference, decision making mechanism, political party influence, partisan alignment, fiscal federalism, principal components analysis (PCA), panel data analysis, grant allocation,mechanisms, tax collection effort, Ramsey model, kalman filter}


\vspace{-0.3in}
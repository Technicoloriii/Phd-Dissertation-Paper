% Place abstract below.
How central government and subnational government share funding and administrative responsibilities has always been a topical and difficult issue for most sovereign states in public administration and public finance area. Introduced by the German-born American economist Musgrave\cite{musgrave1971economics}, fiscal federalism has been massively investigated as a general framework to deal with the issue of funding and responsibilities in both the unitary and federal systems. The field of fiscal federalism studies how to divide responsibilities (including finances) between central and subnational governments to supply public goods and services with economic efficiency and achieve various public policy objectives. Determining the optimal division of responsibilities is difficult because of varying subjective views about what the role of government should be. As a result, fiscal federalism research generally renders no judgment on the proper level of total government intervention or what types of services governments should provide. The research focuses instead on how responsibilities are assigned across multiple layers of government once policymakers have decided to implement a given policy, and what trade-offs may be involved in administering it.	

In this paper, I analyzed the framework of fiscal federalism by separately investigating four questions in terms of both central and subnational level governments. By analyzing the theoretical design and practical situation, I generate a overall understanding about the general fiscal federalism setting, explained the gap between theoretical design and actual administration process theoretically and empirically tested if the theoretical inference is reliable. TO be more specific, by comparatively analyzing the fiscal structure design and administrative reality in United States and China, I investigated the role of political intention in causing the difference, and explained theoretically why the administrative reality and theoretical design are different in both countries under the asymmetric setting using game theory tools. Besides, I also did some empirical test to statistically support some of the theoretical inference.


Key words: fiscal federalism, asymmetric game theory analysis, comparative study


\vspace{-0.3in}